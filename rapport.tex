\documentclass[a4paper]{article}

\usepackage[french]{babel}
\usepackage[utf8]{inputenc}
\usepackage{amsmath}
\usepackage{graphicx}
\usepackage[colorinlistoftodos]{todonotes}
\usepackage{url}

\title{Rapport ADR : Reconnaissance de documents pour la réalité augmentée}

\author{Gwendal Le Moulec}

\date{\today}

\begin{document}
\maketitle

\begin{abstract}
Your abstract.
\end{abstract}

\section{Introduction}

A l'ère de l'information à grande échelle, la manière de lire de documents n'est plus la même qu'il y a vingt ans. La grande disponibilité des multitudes de sources de données permet à n'importe quel lecteur d'élargir un sujet de lecture, en cherchant lui-même les documents "liés", c'est à dire les documents qui traitent d'au moins une information contenue dans le document initial. Bien souvent, de simples mots-clés couvrent un sujet connexe au sujet principal. Ces mots-clés sont suffisants pour définir un critère de recherche dans une base de données.

Un champ important de recherche en informatique consiste à automatiser ce suivi de liens dans les documents. Les hyperliens que l'on trouve dans les bases de documents comme la fameuse encyclopédie en ligne Wikipédia matérialisent ces connexions de manière simple. Cependant, ce genre d'approche présente quelques faiblesses~: elle ne s'applique qu'au documents numériques et c'est l'auteur du document qui doit définir le lien physique ou au moins préparer le document de manière à ce que la définition du lien soit possible \textit{a posteriori}. Ainsi, un document rédigé dans un format non compatible ne peut définitivement plus être à l'origine de liens.

Cette situation est un peu absurde, puisque ce n'est pas le format du document qui apporte la connaissance, mais bien le contenu du document lui-même. C'est pourquoi on recherche aujourd'hui des méthodes qui permettent d'indexer automatiquement les contenus et bien sûr d'interagir avec des documents indépendamment de leur format, qu'ils soient numériques ou non. Les auteurs des articles présentés dans le présent rapport proposent un mécanisme basé sur la réalité augmentée. La réalité augmentée est un paradigme qui couvre les technologies permettant d'ajouter des données virtuelles dans un monde réel et d'introduire de l'interaction entre les deux \cite{augmented-reality}. Concrètement, le système présenté est constitué de lunettes de réalité augmentée qui permettent d'afficher du contenu virtuel lié au mot fixé par le lecteur. Pour ce faire, le système tente d'abord de reconnaître le document parmi une multitude de références dans une base de données. Ensuite, la reconnaissance d'un mot fixé par le lecteur se fait par mise en relation entre la position du point fixé par le lecteur - détectable grâce à un dispositif de suivi oculaire - et le mot situé à cette position dans la version référencée du document.

Nous présenterons tout d'abord les deux articles proposés afin d'en faire un synthèse. Ils feront ensuite l'objet d'une critique mettant en avant les qualités et les défauts des explications, de la démarche et des résultats présentés.

\section{Présentation des articles}



\newpage

\section{Some \LaTeX{} Examples}
\label{sec:examples}

\subsection{How to Leave Comments}

Comments can be added to the margins of the document using the \todo{Here's a comment in the margin!} todo command, as shown in the example on the right. You can also add inline comments:

\todo[inline, color=green!40]{This is an inline comment.}

\subsection{How to Include Figures}

First you have to upload the image file (JPEG, PNG or PDF) from your computer to writeLaTeX using the upload link the project menu. Then use the includegraphics command to include it in your document. Use the figure environment and the caption command to add a number and a caption to your figure. See the code for Figure \ref{fig:frog} in this section for an example.

%\begin{figure}
%\centering
%\includegraphics[width=0.3\textwidth]{frog.jpg}
%\caption{\label{fig:frog}This frog was uploaded to writeLaTeX via the project menu.}
%\end{figure}

\subsection{How to Make Tables}

Use the table and tabular commands for basic tables --- see Table~\ref{tab:widgets}, for example.

\begin{table}
\centering
\begin{tabular}{l|r}
Item & Quantity \\\hline
Widgets & 42 \\
Gadgets & 13
\end{tabular}
\caption{\label{tab:widgets}An example table.}
\end{table}

\subsection{How to Write Mathematics}

\LaTeX{} is great at typesetting mathematics. Let $X_1, X_2, \ldots, X_n$ be a sequence of independent and identically distributed random variables with $\text{E}[X_i] = \mu$ and $\text{Var}[X_i] = \sigma^2 < \infty$, and let
$$S_n = \frac{X_1 + X_2 + \cdots + X_n}{n}
      = \frac{1}{n}\sum_{i}^{n} X_i$$
denote their mean. Then as $n$ approaches infinity, the random variables $\sqrt{n}(S_n - \mu)$ converge in distribution to a normal $\mathcal{N}(0, \sigma^2)$.

\subsection{How to Make Sections and Subsections}

Use section and subsection commands to organize your document. \LaTeX{} handles all the formatting and numbering automatically. Use ref and label commands for cross-references.

\subsection{How to Make Lists}

You can make lists with automatic numbering \dots

\begin{enumerate}
\item Like this,
\item and like this.
\end{enumerate}
\dots or bullet points \dots
\begin{itemize}
\item Like this,
\item and like this.
\end{itemize}
\dots or with words and descriptions \dots
\begin{description}
\item[Word] Definition
\item[Concept] Explanation
\item[Idea] Text
\end{description}

We hope you find write\LaTeX\ useful, and please let us know if you have any feedback using the help menu above.


\begin{thebibliography}{1}
	\bibitem{augmented-reality} Augmented Reality, "C'est quoi la Réalité Augmentée ?", [En ligne]. Accessible~: \url{www.augmented-reality.fr/cest-quoi-la-realite-augmentee/}. [Accès le 28 Octobre 2014].
\end{thebibliography}

% biblio
% http://www.augmented-reality.fr/cest-quoi-la-realite-augmentee/

\end{document}